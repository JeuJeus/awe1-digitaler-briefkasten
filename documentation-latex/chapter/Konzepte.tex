%!TEX root = ../Thesis.tex


\section{Konzepte \textcolor{blue}{[Julius Figge]}}

\subsection{Arbeitskonzept}
\label{Arbeitskonzept}
Unsere Teamarbeit haben wir auf den Austausch untereinander ausgerichtet.
So konnten wir uns unsere unterschiedlichen Kompetenzen zu Nutze machen und haben besipielhaft im Mob-Programming\footnote{Hiermit ist das gemeinse Programmieren über ein Videotelefonat gemeint, bei dem abwechselnd eine Person den Bildschirm teilt.} Wissen vermittelt und uns gegenseitig unterstützt.\\
Darüber hinaus haben wir nach dem \enquote{All hands on Deck}-Prinzip\footnote{Dieses bezeichnet den Ansatz, bei auftretenden Problemen und Fehlern sich zuerst auf die Behebung dieser zu konzentrieren, bevor weitergehende Aufgaben bearbeitet werden.} gearbeitet.\\
Das haben wir zum einen aufbauend auf teaminterne Kommunikation (beispielhaft über Telegram), aber insbesondere auch über die \enquote{Github CI} erreicht.
Diese war konfiguriert bei jedem Commit alle Tests durchzuführen und bei Problemen Email-Benachrichtigungen zu versenden.\\
Ausserdem haben wir \enquote{Sonarlint} eingesetzt um unsere Code-Qualität zu überprüfen und stetig zu verbessern.
Dadurch haben wir nicht nur unsere Code-Qualität sichergestellt, sondern konnten auch auftretenden Probleme möglichst schnell erkennen und beheben.\\
Hierbei ist positiv anzumerken, dass durch das gemeinsame Arbeiten Wissensilos effektiv aufgebrochen wurden und der Lerneffekt im Zuge des Projekts für alle Beteiligten maximiert wurde.

\subsection{MVC-Pattern}
In unserer Anwendung benutzen wir das Architekturmuster Model View Controller.
Dieses Muster haben wir explizit ausgewählt, da Springboot zusammen mit Thymeleaf als Frontend hierfür sehr gut geeignet ist. Dadurch erreichen wir eine strikte Trennung der verschiedenen Ebenen und eine bessere zukünftige Anpassbarkeit des Programmes.

\subsection{Jackson-JSON}
Wir haben uns für die Nutzung der \enquote{Jackson}-JSON library entschieden unter anderem, da diese Annotations mitliefert, welche wir direkt in unseren Code einbinden können. Diese benutzen wir um die Serealisierung und Deserealisierung von Datenbank-Einträgen zu konfigurieren. Dadurch ist es einfach im Code zu kontrollieren welche Einträge wie serealisiert werden. Das ist insbesondere für die Entwicklung der API relevant.