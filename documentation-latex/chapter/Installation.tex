%!TEX root = ../Thesis.tex


\section{Installation \textcolor{blue}{[Jonathan Brockhausen]}}
\label{instal}
Das Programm setzt eine installierte Java Runtime Environment der Version 11 voraus.
\subsubsection*{Kompilieren und Starten des Programmes}
\textit{Wenn bereits eine JAR vorhanden ist, kann direkt zu Punkt 2 gegangen werden.}
\begin{enumerate}
	\item{Das Programm kann mithilfe des Maven Wrappers kompiliert werden:}
	\begin{enumerate}
		\item{unter Linux / MacOS Systemen mit: \texttt{chmod +x mvnw \& ./mvnw clean compile compile package}}
		\item{unter Windows mit: \texttt{mvnw.cmd clean compile package}}
	\end{enumerate}
	\item{Danach kann die Zieldatei aus dem Projekt Root-Verzeichnis ausgeführt werden.}
	\begin{enumerate}
		\item{\texttt{java -jar target/digitaler-briefkasten-1.0.1-ABGABE.jar} (Die Versionsnummer kann abweichen!)}
	\end{enumerate}
	\item{Nach dem erfolgreichen Start ist die Oberfläche unter \texttt{http://localhost:8080} erreichbar.}
\end{enumerate}

\subsubsection*{Test-Zugangsdaten}
Grundsätzlich existieren drei verschiedene Arten von Accounts:
\begin{enumerate}
	\item{Administrator}
	\item{Spezialist}
	\item{User}
\end{enumerate}

Zur Nutzung des Systems als User kann ein neuer User-Account registriert werden.\\
Ein Administrator-Account und ein Spezialisten-Account können mithilfe der Methoden in der Klasse \texttt{HelperScriptsNoTests} angelegt werden. Die Zugangsdaten sind wie folgt:\\

\textbf{Administrator}
\begin{enumerate}
	\item{\textbf{Username:} admin}
	\item{\textbf{Passwort:} hierKönnteIhreWerbungStehen}
\end{enumerate}
\pagebreak
\textbf{Spezialist}
\begin{enumerate}
	\item{\textbf{Username:} SpeziusMaximus\textunderscore[Bezeichnung der jeweiligen Produktsparte]}
	\item[]{Beispiel: SpeziusMaximus\textunderscore{}INTERNAL}
	\item{\textbf{Passwort:} boringProphet}
\end{enumerate}
