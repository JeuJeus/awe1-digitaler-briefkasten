%!TEX root = ../Thesis.tex


\section{Test \textcolor{blue}{[Julius Figge]}}

\subsection{manuelle-\enquote{Klicktests}}

Zur Überprüfung der \enquote{GUI} sollen manuelle Klicktests durchgeführt werden.
Diese sollen dokumentiert werden um Fehler möglichst gezielt beheben zu können.\\

\subsubsection*{Zu notierende Informationen}
Zu den notierenden Informationen gehören zum einen die Programmrevision (Git Commit Hash, Datum) sowie der verwendete Branch. Darüber hinaus ist das genutzte Betriebssystem sowie der genutzte Browser (inklusive Build zu notieren).
Bei Darstellungsfehlern ist es sinnvoll zudem Screenshots zu hinterlegen sowie die Bildschirmauflösung zu notieren.
Diese Informationen sammeln wir gezielt sehr detailliert um Fehler besser eingrenzen zu können.

\subsubsection*{Testvorbereitung}
\begin{enumerate}
    \item Zum Testen wird der neueste Stand des Master-Branches verwendet.
    \item Hierzu ist zunächst die Datenbank zu löschen und mit Hilfe der in \enquote{HelperScriptsNoTests}        vorhandenen Tests zu füllen.
    \item Der Code soll kompiliert werden und die entstandene \enquote{Jar}-Datei ausgeführt werden.
    \item Nach Möglichkeit soll der Test auf mehreren Browsern ausgeführt werden. Hierbei ist zu beachten, dass alle Addons zu deaktivieren sind, um eventuelle Komplikationen auszuschließen.
    \item Nachdem diese Voraussetzung geschaffen ist, sind die Tests durchzuführen und die obigen Informationen zu notieren.
\end{enumerate}

Zur Testdurchführung ist die Tabelle im Anhang \ref{fig:testdurchf} auf S.\pageref{fig:testdurchf} zu verwenden.

