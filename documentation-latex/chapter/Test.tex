%!TEX root = ../Thesis.tex


\section{Test}
\subsection{manuelle-\enquote{Klicktests}}

Zur Überprüfung der \enquote{GUI} sollen manuelle Klicktests durchgeführt werden.
Diese sollen dokumentiert werden um Fehler möglichst gezielt beheben zu können.\\

\subsubsection*{Zu notierende Informationen}
\begin{itemize}
\item Revision (Git Commit Hash, Datum) sowie Branch
\item Betriebssystem
\item Browser (inklusive Build)
\item Bildschirmauflösung, Fenstergröße (nach Möglichkeit)
\item auftretende Fehler (inklusive Screenshots)
\end{itemize}

\subsubsection*{Testvorbereitung}
\begin{enumerate}
    \item Zum Testen wird der neueste Stand des master-Branches verwendet. 
	\item Hierzu ist zunächst die Datenbank zu löschen und mit Hilfe der in \enquote{HelperScriptsNoTests} 		vorhandenen Tests zu füllen.
	\item Der Code soll kompiliert werden und die entstandene \enquote{Jar}-Datei ausgeführt werden.
    \item Nach Möglichkeit soll der Test auf mehreren Browsern ausgeführt werden. Hierbei ist zu beachten, dass alle Addons zu deaktivieren sind, um eventuelle Komplikationen auszuschließen.
    \item Nachdem diese Voraussetzung geschaffen ist, sind die Tests durchzuführen und die obigen Informationen zu notieren.
\end{enumerate}

\subsubsection*{Testdurchführung}

\begin{center}
\begin{longtable}{|p{0.4\textwidth}|p{0.3\textwidth}|p{0.2\textwidth}|}
\caption{GUI-Testdurchführung}\\
\hline
Aktion & erwartetes Ergebnis & Reaktion\\
\hline
\hline

\textbf{Registrieren eines neuen Nutzers} & &\\
\hline
Bereits bestehenden Nutzernamen verwenden (admin) & Fehlermeldung - Nutzername existiert bereits &\\
\hline
zu kurzer Nutzername (<3) & Fehlermeldung - Daten falsch &\\
\hline
zu kurzes Passwort (<=7) & Fehlermeldung - zu kurzes Passwort &\\
\hline
nicht übereinstimmende Passwörter & Fehlermeldung - nicht stimmende Passwort &\\
\hline
mit korrekten Daten & eingeloggt sein &\\
\hline
\hline

\textbf{Ausloggen aus dem Account} & ausgeloggt sein &\\
\hline
\hline

\textbf{Einloggen in erstellten Account} & &\\
\hline
mit falschem Passwort & Fehlermeldung &\\
\hline
mit falschem Nutzernamen & Fehlermeldung &\\
\hline
mit richtigem Passwort & eingeloggt sein &\\
\hline
\hline

\textbf{Erstellen von beispielhaften Ideen} & &\\
\hline
Erstellen einer \enquote{internen Idee} & Idee erscheint in Tabelle nicht eingereichter Ideen &\\
\hline
Erstellen einer \enquote{Produkt-Idee} & Idee erscheint in Tabelle nicht eingereichter Ideen &\\
\hline
Erstellen einer beliebigen Idee mit fehlerhaften Werten & Fehlerhafte Attribute werden hervorgehoben &\\
\hline
Erstellen einer Idee von der bereits selber Name bei selbem Typ vorhanden & Fehlermeldung über Duplikat &\\
\hline
Bearbeitend der internen Idee & Änderungen werden übernommen &\\
\hline
Bearbeitend der Produkt-Idee & Änderungen werden übernommen &\\
\hline
\hline

\textbf{Ideenübersicht} & &\\
\hline
Filtern der nicht eingereichten Ideen nach Attributen & nur Ideen mit passenden Attributen werden angezeigt &\\
\hline 
Einreichen der erstellten Ideen & erfolgreicher Transfer in jeweilige Tabelle &\\
\hline
\hline

\textbf{Ausloggen aus dem Account} & ausgeloggt sein &\\
\hline
\hline

\textbf{Idee Übersicht als nicht eingeloggter Nutzer} & &\\
\hline 
Filtern der Ideen in beiden Tabellen & nur Ideen mit passenden Attributen werden angezeigt &\\
\hline
\hline

\textbf{Spezialist für \enquote{internen Idee}} & &\\
\hline
Einloggen als passender (Ideen sollten ihm zugewiesen sein) Spezialist (Zugangsdaten siehe \texttt{Manual.md})& eingeloggt sein &\\
\hline
Übersicht zu entscheidender Ideen filtern &  nur Ideen mit passenden Attributen werden angezeigt &\\
\hline
Entscheiden ohne Begründung & fehlendes Attribut wird hervorgehoben &\\
\hline
Idee in Ideenspeicher verschieben & Idee liegt in Ideenspeicher &\\
\hline
\hline

\textbf{Spezialist für \enquote{Produkt-Idee}} & &\\
\hline
Account zu anderem Spezialist wechseln & Idee liegt in Ideenspeicher &\\
\hline
Entscheiden über Idee aus Ideenspeicher mit Auswahl  \enquote{zur Entscheidung freigegeben} & Idee liegt in eigenen zu entscheidenden Ideen &\\
\hline
Idee aus Entscheidungsübersicht bewerten & Idee erscheint auf passender Tabelle in Ideenübersicht &\\
\hline
\hline

\textbf{Administrator} & &\\
\hline
Account zu Administrator wechseln (Zugangsdaten siehe \texttt{Manual.md})& &\\
\hline
Existierende User prüfen & registrierter Account sowie alle Spezialisten werden aufgelistet &\\
\hline 
Alle möglichen anzulegenden Felder durchgehen, bereits bestehenden Namen eingeben & bei jedem Feld wird ein Fehler angezeigt &\\
\hline 
Alle möglichen anzulegenden Felder durchgehen & Feld wird angelegt &\\
\hline
Ausloggen & \textbf{FERTIG!} &\\
\hline
\end{longtable}
\end{center}
