%!TEX root = ../Thesis.tex
\section*{Anhang}
\addcontentsline{toc}{section}{Anhang}
\fancyhead[R]{Anhang}

\anhangsverzeichnis

\anhang{Testdurchführung \textcolor{blue}{[Julius Figge]}}

\begin{center}
    \label{fig:testdurchf}
    \begin{longtable}{|p{0.45\textwidth}|p{0.35\textwidth}|p{0.1\textwidth}|}
        \caption{GUI-Testdurchführung}\\
        \hline
        Aktion & erwartetes Ergebnis & Reaktion\\
        \hline
        \hline

        \textbf{Registrieren eines neuen Nutzers} & &\\
        \hline
        Bereits bestehenden Nutzernamen verwenden (admin) & Fehlermeldung - Nutzername existiert bereits & \ding{51}\\
        \hline
        zu kurzer Nutzername (<3) & Fehlermeldung - Daten falsch & \ding{51}\\
        \hline
        zu kurzes Passwort (<=7) & Fehlermeldung - zu kurzes Passwort & \ding{51}\\
        \hline
        nicht übereinstimmende Passwörter & Fehlermeldung - nicht stimmende Passwort & \ding{51}\\
        \hline
        mit korrekten Daten & eingeloggt sein & \ding{51}\\
        \hline
        \hline

        \textbf{Ausloggen aus dem Account} & ausgeloggt sein & \ding{51}\\
        \hline
        \hline

        \textbf{Einloggen in erstellten Account} & &\\
        \hline
        mit falschem Passwort & Fehlermeldung & \ding{51}\\
        \hline
        mit falschem Nutzernamen & Fehlermeldung & \ding{51}\\
        \hline
        mit richtigem Passwort & eingeloggt sein & \ding{51}\\
        \hline
        \hline

        \textbf{Erstellen von beispielhaften Ideen} & &\\
        \hline
        Erstellen einer \enquote{internen Idee} & Idee erscheint in Tabelle nicht eingereichter Ideen & \ding{51}\\
        \hline
        Erstellen einer \enquote{Produkt-Idee} & Idee erscheint in Tabelle nicht eingereichter Ideen & \ding{51}\\
        \hline
        Erstellen einer beliebigen Idee mit fehlerhaften Werten & Fehlerhafte Attribute werden hervorgehoben & \ding{51}\\
        \hline
        Erstellen einer Idee von der bereits selber Name bei selbem Typ vorhanden & Fehlermeldung über Duplikat &\ding{51}\\
        \hline
        Bearbeiten der internen Idee & Änderungen werden übernommen & \ding{51}\\
        \hline
        Bearbeiten der Produkt-Idee & Änderungen werden übernommen & \ding{51}\\
        \hline
        \hline

        \textbf{Ideenübersicht} & &\\
        \hline
        Filtern der nicht eingereichten Ideen nach Attributen & nur Ideen mit passenden Attributen werden angezeigt & \ding{51}\\
        \hline
        Einreichen der erstellten Ideen & erfolgreicher Transfer in jeweilige Tabelle & \ding{51}\\
        \hline
        \hline

        \textbf{Ausloggen aus dem Account} & ausgeloggt sein & \ding{51}\\
        \hline
        \hline

        \textbf{Idee Übersicht als nicht eingeloggter Nutzer} & &\\
        \hline
        Filtern der Ideen in beiden Tabellen & nur Ideen mit passenden Attributen werden angezeigt & \ding{51}\\
        \hline
        \hline

        \textbf{Kontaktformular} & & \\
        \hline
        Kontaktformular im Footer aufrufen & Kontaktformular erscheint & \ding{51}\\
        \hline
        Nachricht mit Titel, Nachricht und eigener E-Mail-Adresse absenden & Nachricht abgesendet & \ding{51}\\
        \hline
        Nachricht mit Titel, Nachricht und falsch formatierter E-Mail-Adresse absenden & Fehler & \ding{51}\\
        \hline
        \hline

        \textbf{Spezialist für \enquote{internen Idee}} & &\\
        \hline
        Einloggen als passender (Ideen sollten ihm zugewiesen sein) Spezialist (Zugangsdaten siehe \texttt{Manual.md})& eingeloggt sein & \ding{51}\\
        \hline
        Übersicht zu entscheidender Ideen filtern & nur Ideen mit passenden Attributen werden angezeigt & \ding{51}\\
        \hline
        Entscheiden ohne Begründung & fehlendes Attribut wird hervorgehoben & \ding{51}\\
        \hline
        Idee in Ideenspeicher verschieben & Idee liegt in Ideenspeicher & \ding{51}\\
        \hline
        \hline

        \textbf{Spezialist für \enquote{Produkt-Idee}} & &\\
        \hline
        Account zu anderem Spezialist wechseln & Eingeloggt und Idee liegt in Ideenspeicher & \ding{51}\\
        \hline
        Entscheiden über Idee aus Ideenspeicher mit Auswahl  \enquote{zur Entscheidung freigegeben} & Idee liegt in eigenen zu entscheidenden Ideen & \ding{51}\\
        \hline
        Idee aus Entscheidungsübersicht bewerten & Idee erscheint auf passender Tabelle in Ideenübersicht & \ding{51}\\
        \hline
        Ausloggen & Ausgeloggt und angenommene Idee mit Begründung in Ideenübersicht & \ding{51}\\
        \hline
        \hline

        \textbf{Administrator} & &\\
        \hline
        Account zu Administrator wechseln (Zugangsdaten siehe \texttt{Manual.md})& Eingeloggt auf Admin-Seite & \ding{51}\\
        \hline
        Existierende User anschauen & registrierter Account sowie alle Spezialisten werden aufgelistet & \ding{51}\\
        \hline
        Neuen Fachspezialisten anlegen & Spezialist taucht in Userliste auf & \ding{51}\\
        \hline
        Neue Produktsparte anlegen & Produktsparte angelegt & \ding{51}\\
        \hline
        Neue Handlungsfeld anlegen & Handlungsfeld angelegt & \ding{51}\\
        \hline
        Neue Zielgruppe anlegen & Zielgruppe angelegt & \ding{51}\\
        \hline
        Neue Vertriebskanal anlegen & Vertriebskanal angelegt & \ding{51}\\
        \hline
        Vertriebskanal mit dem selben Titel anlegen & Fehler wird angezeigt & \ding{51}\\
        \hline
        Kontaktnachricht ansehen und als beantwortet kennzeichnen & Keine ungelesenen Nachrichten & \ding{51}\\
        \hline
        Ausloggen & Ausgeloggt & \ding{51}\\
        \hline
    \end{longtable}
\end{center}

\clearpage
\pagebreak

\anhang{Weitere Use-Cases \textcolor{blue}{[Julius Figge]}}\label{Anhang-Use-Cases}

\subanhang{Administrator}\label{Anhang-Admin}
\begin{figure}[h]
    \centering
    \begin{minipage}[t]{1\textwidth}
        \caption{Administrator - Use-Case Diagramm}
        \includegraphics[width=1\textwidth]{img/admin-use-case.png}\\
        \source{Eigene Darstellung}
    \end{minipage}
\end{figure}

\subanhang{Kontaktformular}\label{Anhang-Kontakt}
\begin{figure}[h]
    \centering
    \begin{minipage}[t]{1\textwidth}
        \caption{Kontaktformular - Use-Case Diagramm}
        \includegraphics[width=1\textwidth]{img/kontakt-use-case.png}\\
        \source{Eigene Darstellung}
    \end{minipage}
\end{figure}

\clearpage
\pagebreak

\anhang{GUI-Konzept \textcolor{blue}{[Julius Figge]}}
\label{GUI-Konzept}
\subanhang{Konzept}

\begin{figure}[hbt]
    \centering
    \begin{minipage}[t]{1\textwidth}
        \caption{GUI-Konzept - Login}
        \includegraphics[width=1\textwidth]{img/login-konzept.png}\\
        \source{Eigene Darstellung}
        \label{fig:login}
    \end{minipage}
\end{figure}

\begin{figure}[h]
    \centering
    \begin{minipage}[t]{1\textwidth}
        \caption{GUI-Konzept - Registrierung }
        \includegraphics[width=1\textwidth]{img/registrierung-konzept.png}\\
        \source{Eigene Darstellung}
    \end{minipage}
\end{figure}

\begin{figure}[h]
    \centering
    \begin{minipage}[t]{1\textwidth}
        \caption{GUI-Konzept - Willkommen}
        \includegraphics[width=1\textwidth]{img/welcome-konzept.png}\\
        \source{Eigene Darstellung}
    \end{minipage}
\end{figure}

\begin{figure}[h]
    \centering
    \begin{minipage}[t]{1\textwidth}
        \caption{GUI-Konzept - Idee erstellen}
        \includegraphics[width=1\textwidth]{img/createIdea-konzept.png}\\
        \source{Eigene Darstellung}
    \end{minipage}
\end{figure}

\clearpage
\pagebreak

\subanhang{Umsetzung}\label{GUI-Umsetzung}

Die im folgenden dargestellten GUI Bestandteile stellen die wichtigsten Teile der Oberfläche dar. Auf die Abbildung aller Bestandteile wurde aufgrund der zu großen Menge, zur Wahrung der Übersichtlichkeit, verzichtet.

\begin{figure}[h]
    \centering
    \begin{minipage}[t]{1\textwidth}
        \caption{GUI-Umsetzung - Login }
        \includegraphics[width=1\textwidth]{img/login-umsetzung.png}\\
        \source{Eigene Darstellung}
    \end{minipage}
\end{figure}

\begin{figure}[h]
    \centering
    \begin{minipage}[t]{1\textwidth}
        \caption{GUI-Umsetzung - Registrierung }
        \includegraphics[width=1\textwidth]{img/registrierung-umsetzung.png}\\
        \source{Eigene Darstellung}
    \end{minipage}
\end{figure}

\begin{figure}[h]
    \centering
    \begin{minipage}[t]{1\textwidth}
        \caption{GUI-Umsetzung - Ideen}
        \includegraphics[width=1\textwidth]{img/ideen-umsetzung.png}\\
        \source{Eigene Darstellung}
    \end{minipage}
\end{figure}

\begin{figure}[h]
    \centering
    \begin{minipage}[t]{1\textwidth}
        \caption{GUI-Umsetzung - Idee erstellen }
        \includegraphics[width=1\textwidth]{img/createIdea-umsetzung.png}\\
        \source{Eigene Darstellung}
    \end{minipage}
\end{figure}

\begin{figure}[h]
    \centering
    \begin{minipage}[t]{1\textwidth}
        \caption{GUI-Umsetzung - Idee ansehen }
        \includegraphics[width=1\textwidth]{img/idee-umsetzung.png}\\
        \source{Eigene Darstellung}
    \end{minipage}
\end{figure}

\begin{figure}[h]
    \centering
    \begin{minipage}[t]{1\textwidth}
        \caption{GUI-Umsetzung - Admin Ansicht }
        \includegraphics[width=1\textwidth]{img/admin-umsetzung.png}\\
        \source{Eigene Darstellung}
    \end{minipage}
\end{figure}

\begin{figure}[h]
    \centering
    \begin{minipage}[t]{1\textwidth}
        \caption{GUI-Umsetzung - Spezialist Ansicht }
        \includegraphics[width=1\textwidth]{img/spezialist-umsetzung.png}\\
        \source{Eigene Darstellung}
    \end{minipage}
\end{figure}

\pagebreak
\clearpage

\anhang{Projektplanung}
\subanhang{Projektstrukturplan}
\label{PSP}
\begin{figure}[h]
    \centering
    \begin{minipage}[t]{1\textwidth}
        \caption{Projektstrukturplan}
        \includegraphics[width=1\textwidth]{img/PSP.png}\\
        \source{Eigene Darstellung}
    \end{minipage}
\end{figure}

\pagebreak
\clearpage

\subanhang{Soll-Ist-Vergleich Muss- und Kann-Features}
\label{PM_SOLLIST}
\begin{center}
    \begin{tabularx}{\linewidth}{
        |p{\dimexpr.09\linewidth-2\tabcolsep-1.3333\arrayrulewidth}% column 2
        |p{\dimexpr.66\linewidth-2\tabcolsep-1.3333\arrayrulewidth}% column 1
        |p{\dimexpr.25\linewidth-2\tabcolsep-1.3333\arrayrulewidth}|% column 3
    }
        \hline
        & Anforderung & Umsetzung \\ \hline
        Muss & Noch nicht registrierte Mitarbeiter können sich am System registrieren & Umgesetzt \\ \hline
        Muss & Registrierte Mitarbeiter können sich am System anmelden & Umgesetzt \\ \hline
        Muss & Registrierte Mitarbeiter können neue Ideen erfassen & Umgesetzt \\ \hline
        Muss & Registrierte Mitarbeiter können sich eine Liste ihrer eingereichten Ideen anzeigen lassen & Umgesetzt \\ \hline
        Muss & Registrierte Mitarbeiter können ihre Ideen solange bearbeiten oder auch löschen solange dieses noch nicht zur Bewertung an einen Fachspezialisten übergeben wurden. & Umgesetzt \\ \hline
        Muss & Nicht registrierte Mitarbeiter können vorhandene Ideen lesen, sich eine Übersicht der Ideen anzeigen lassen und die Übersicht filtern & Umgesetzt \\ \hline
        Muss & Diese Funktionen stehen auch registrierten Mitarbeitern zur Verfügung & Umgesetzt \\ \hline
        Muss & Neue Ideen werden Fachspezialisten zur Bewertung zugeordnet & Umgesetzt \\ \hline
        Muss & Die Zuordnung erfolgt automatisch sobald die Idee vom registrierten Mitarbeiter zur Bewertung eingereicht wurde & Umgesetzt \\ \hline
        Muss & Fachspezialisten können eine Idee entweder annehmen, ablehnen oder für einen späteren Zeitpunkt in einen sog. Ideenspeicher überführen / sie aus dem Ideenspeicher zurückholen & Umgesetzt \\ \hline
        Muss & Fachspezialisten begründen ihre Entscheidung transparent und für alle sichtbar in der Anwendung & Umgesetzt \\ \hline
        Muss & Fachspezialisten können ihnen zugewiesene Ideen in einer Liste sehen und diese Liste filtern & Umgesetzt \\ \hline
    \end{tabularx}
\end{center}
\pagebreak
\begin{center}
    \begin{tabularx}{\linewidth}{
        |p{\dimexpr.1\linewidth-2\tabcolsep-1.3333\arrayrulewidth}% column 2
        |p{\dimexpr.45\linewidth-2\tabcolsep-1.3333\arrayrulewidth}% column 1
        |p{\dimexpr.45\linewidth-2\tabcolsep-1.3333\arrayrulewidth}|% column 3
    }
        \hline
        & Anforderung & Umsetzung \\ \hline
        Kann & REST-API & Teilweise umgesetzt, lauffähig und erweiterbar \\ \hline
        Kann & Kontaktformular auch unregistriert & Umgesetzt, erweiterbar um E-Mail-Einbindung \\ \hline
        Kann & Administrator verwaltet Benutzer & Umgesetzt, erweiterbar \\ \hline
        Kann & Dokumentenupload zu einer Idee & Nicht umgesetzt, mit Erweiterung der Datenbank umsetzbar \\ \hline
        Kann & Profilfoto & Nicht umgesetzt, mit Erweiterung der Datenbank umsetzbar \\ \hline
        Kann & Fachspezialist: E-Mail Benachrichtigung bei neuer Idee & Nicht umgesetzt, erfordert E-Mail-Einbindung \\ \hline
        Kann & Benutzer: E-Mail Benachrichtigung bei Änderung einer Idee & Nicht umgesetzt, erfordert E-Mail-Einbindung \\ \hline
        Kann & PDF-Report über erstellte Ideen quartalsweise & Nicht umgesetzt \\ \hline
    \end{tabularx}
\end{center}

\pagebreak
\clearpage
\anhang{Schnittstellen \textcolor{blue}{[Philipp Röring]}}
\label{Anhang:Schnittstellen}
\subanhang{Antwort /api/ideas/}
\label{Anhang:Schnittstellen1}
%TODO ANNOTATE ME AS LISTING TO BE ADDED TO LISTING LISTING
\begin{lstlisting}[language=json, caption=Antwort /api/ideas/, label=list:schnittstellen1]
[
        {
                "id": 145,
        "title": "Nachmieter für Häuschen in Detmold gesucht!",
        "description": "Och joa ich habe da 'ne Idee",
        "creator": {
                "type": "de.fhdw.geiletypengmbh.digitalerbriefkasten.
        persistance.model.account.User",
        "id": 1,
        "username": "API_USER",
        "roles": [
                {
                        "name": "API_USER"
                }
        ],
        "lastName": "USER",
        "firstName": "API",
        "creationDate": "2020-05-24"
        },
    "creationDate": "2020-05-27",
    "status": "NOT_SUBMITTED",
    "productLine": {
                "id": 2,
        "title": "INTERNAL",
        "specialists": []
        },
    "advantages": [
                {
                        "id": 146,
            "description": "Nur"
                },
                {
                        "id": 147,
            "description": "Ein"
                },
                {
                        "id": 148,
            "description": "Vorteil"
                }
        ],
    "specialist": null,
    "field": {
                "id": 21,
        "title": "Kostensenkung"
        }
        },
        {
                "id": 149,
        "title": "[Reserviert] Nachmieter für Häuschen in Detmold gesucht!",
    "description": "Hmmm.. Naja irgendwas wird es schon werden.",
    "creator": {
                "type": "de.fhdw.geiletypengmbh.digitalerbriefkasten.
        persistance.model.account.User",
        "id": 1,
        "username": "API_USER",
        "roles": [
                {
                        "name": "API_USER"
                }
        ],
        "lastName": "USER",
        "firstName": "API",
        "creationDate": "2020-05-24"
        },
    "creationDate": "2020-05-27",
    "status": "NOT_SUBMITTED",
    "productLine": {
                "id": 3,
        "title": "KFZ",
        "specialists": []
        },
    "advantages": [
                {
                        "id": 150,
            "description": ""
                },
                {
                        "id": 151,
            "description": ""
                },
                {
                        "id": 152,
            "description": ""
                }
        ],
    "specialist": null,
    "targetGroups": [
                {
                        "id": 17,
            "title": "Singles"
                }
        ],
    "distributionChannels": [
                {
                        "id": 14,
            "title": "Kooperation mit Kreditinstituten"
                }
        ],
    "existsComparable": true
        }
]
\end{lstlisting}

\subanhang{Antwort /api/ideas/\{id\}}
\label{Anhang:Schnittstellen2}
\begin{lstlisting}[language=json, caption=Antwort /api/ideas/\{id\}, label=list:schnittstellen2]
{
        "type": "de.fhdw.geiletypengmbh.digitalerbriefkasten.
    persistance.model.ideas.ProductIdea",
    "id": 149,
    "title": "[Reserviert] Nachmieter für Häuschen in Detmold gesucht!",
"description": "Hmmm.. Naja irgendwas wird es schon werden.",
"creator": {
        "type": "de.fhdw.geiletypengmbh.digitalerbriefkasten.
    persistance.model.account.User",
    "id": 1,
    "username": "API_USER",
    "roles": [
        {
                "name": "API_USER"
        }
],
    "lastName": "USER",
    "firstName": "API",
    "creationDate": "2020-05-24"
},
"creationDate": "2020-05-27",
"status": "NOT_SUBMITTED",
"productLine": {
        "id": 3,
    "title": "KFZ",
    "specialists": [
        {
                "type": "de.fhdw.geiletypengmbh.digitalerbriefkasten.
            persistance.model.account.Specialist",
            "id": 153,
            "username": "SpeziusMaximus_KFZ",
            "roles": [
                {
                        "name": "SPECIALIST"
                }
        ],
        "lastName": "Maximus",
        "firstName": "Spezius",
        "creationDate": "2020-05-27"
        }
]
},
"advantages": [],
"specialist": null,
"targetGroups": [
        {
                "id": 17,
        "title": "Singles"
        }
],
"distributionChannels": [
        {
                "id": 12,
        "title": "Stationärer Vertrieb"
        },
        {
                "id": 14,
        "title": "Kooperation mit Kreditinstituten"
        }
],
"existsComparable": true
}
\end{lstlisting}
\anhang{Klassendiagramme \textcolor{blue}{[Philipp Röring]}}
\label{Anhang:Klassendiagramme}
\subanhang{Model}

\begin{figure}[htb]
    \centering
    \begin{minipage}[H]{1\textwidth}
        \caption{Klassendiagramm Model Ideen}
        \includegraphics[width=0.7\textwidth]{img/idea-model-klassendiagramm.jpg}\\
        \source{Eigene Darstellung}
        \label{fig:idea-model-klassendiagramm}
    \end{minipage}
\end{figure}

\begin{figure}[htb]
    \centering
    \begin{minipage}[H]{1\textwidth}
        \caption{Klassendiagramm Model User}
        \includegraphics[width=0.55\textwidth]{img/user-model-klassendiagramm.png}\\
        \source{Eigene Darstellung}
        \label{fig:user-model-klassendiagramm}
    \end{minipage}
\end{figure}
