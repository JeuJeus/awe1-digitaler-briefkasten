%!TEX root = ../Thesis.tex


\section{Test}
\subsection{manuelle-\enquote{Klicktests}}

Zur Überprüfung der \enquote{GUI} sollen manuelle Klicktests durchgeführt werden.
Diese sollen dokumentiert werden um Fehler möglichst gezielt beheben zu können.\\

\subsubsection*{Zu notierende Informationen}
\begin{itemize}
\item Revision (Git Commit Hash, Datum) sowie Branch
\item Betriebssystem
\item Browser (inklusive Build)
\item Bildschirmauflösung, Fenstergröße (nach Möglichkeit)
\item auftretende Fehler (inklusive Screenshots)
\end{itemize}

\subsubsection*{Testvorbereitung}
\begin{enumerate}
    \item Zum Testen wird der neueste Stand des master-Branches verwendet. 
	\item Hierzu ist zunächst die Datenbank zu löschen und mit Hilfe der in \enquote{HelperScriptsNoTests} 		vorhandenen Tests zu füllen.
	\item Der Code soll kompiliert werden und die entstandene \enquote{Jar}-Datei ausgeführt werden.
    \item Nach Möglichkeit soll der Test auf mehreren Browsern ausgeführt werden. Hierbei ist zu beachten, dass alle Addons zu deaktivieren sind, um eventuelle Komplikationen auszuschließen.
    \item Nachdem diese Voraussetzung geschaffen ist, sind die Tests durchzuführen und die obigen Informationen zu notieren.
\end{enumerate}

\subsubsection*{Testdurchführung}
\begin{enumerate}
\item Registrieren eines neuen Nutzers
	\subitem Bereits bestehenden Nutzernamen verwenden (admin)
	\subitem Mit nicht den Richtlinien entsprechendem Nutzernamen
	\subitem Mit nicht den Richtlinien entsprechendem Passwort
	\subitem Mit nicht übereinstimmenden Passwörtern
	\subitem Korrektes Passwort
\item Ausloggen aus dem Account
\item Einloggen in den erstellten Account
	\subitem Mit falschem Nutzernamen
	\subitem Mit falschem Passwort
	\subitem Mit richtigem Passwort
\item Erstellen von beispielhaften Ideen
	\subitem Erstellen einer \enquote{internen Idee}
	\subitem Erstellen einer \enquote{Projekt-Idee}
	\subitem Erstellen einer beliebigen Idee mit fehlerhaften / fehlenden Werten
	\subitem Bereits bestehenden Namen verwenden
\item Ideenübersicht prüfen
\item Bearbeiten der Werte beider Ideen
\item Ideenübersicht prüfen
	\subitem Filtern der Tabelle nicht eingereichter Ideen nach nach allen Spalten (und Typ)
\item Einreichen beider Ideen zur Bearbeitung
\item Ideenübersicht prüfen
\item Ausloggen
\item Ansicht der Ideen als nicht eingeloggter Nutzer
	\subitem Filtern der Ideen als nicht eingeloggter Nutzer
\item einloggen als Spezialist für \enquote{internen Idee}
\item Übersicht zu entscheidender Ideen ansehen
	\subitem Übersicht zu entscheidender Ideen filtern
	\subitem Entscheiden über Idee mit fehlerhafter Eingabe
	\subitem Verschieben der Idee in Ideenspeicher
\item Ideenübersicht prüfen
\item Ausloggen
\item Einloggen als Spezialist für vorgehend eingereichte \enquote{Projekt-Idee}
\item Idee aus Ideenspeicher annehmen
\item Idee aus Entscheidungsübersicht bewerten
\item Ideenübersicht prüfen
\item Ausloggen
\item Einloggen als \enquote{admin}
\item Existierende User prüfen
\item Alle möglichen anzulegenden Felder durchgehen
 \subitem Zuerst Bereits bestehenden Namen beim anlegen verwenden
 \subitem Danach neuen Namen verwenden
\item Ausloggen
\item Fertig!
\end{enumerate}
