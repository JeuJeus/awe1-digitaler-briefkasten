%!TEX root = ../Thesis.tex
%! Author = PR
%! Date = 27.05.2020


\section{Schnittstellen \textcolor{blue}{[Philipp Röring]}}

Die Anwendung ist über eine REST-API erreichbar. Alle Antworten sind im JSON-Format.

\subsection{Schnittstellenbeschreibung REST-API}
Die Schnittstelle stellt folgende Services bereit:
\begin{itemize}
    \item HTTP-Methode: GET
    \subitem Relativer Pfad: \textit{/api/ideas/}
    \subitem Antwort: Array, das alle Ideen (Produkt-/ Interne Ideen) beinhaltet.
    \subitem Beispielantwort: siehe Anhang \ref{Anhang:Schnittstellen1}
\end{itemize}

\begin{itemize}
    \item HTTP-Methode: GET
    \subitem Relativer Pfad: \textit{/api/ideas/\{id\}}
    \subitem Antwort: Idee (Produkt-/ Interne Idee) in JSON-Format
    \subitem Beispielantwort: siehe Anhang \ref{Anhang:Schnittstellen2}
\end{itemize}

\begin{itemize}
    \item HTTP-Methode: GET
    \subitem Relativer Pfad: \textit{/api/ideas/title/\{title\}}
    \subitem Antwort: Idee (Produkt-/ Interne Idee) in JSON-Format
    \subitem Beispielantwort: siehe Anhang \ref{Anhang:Schnittstellen2}
\end{itemize}

\begin{itemize}
    \item HTTP-Methode: GET
    \subitem Relativer Pfad: \textit{/api/ideas/submitted}
    \subitem Antwort: Array, das alle Ideen (Produkt-/ Interne Ideen), die veröffentlicht sind, beinhaltet.
    \subitem Beispielantwort: siehe Anhang \ref{Anhang:Schnittstellen1}
\end{itemize}

\begin{itemize}
    \item HTTP-Methode: POST
    \subitem Relativer Pfad: \textit{/api/ideas/}
    \subitem Mitzugebener HTTP-Body: Idee in JSON-Format (Syntax siehe GET-Methoden)
    \subitem Bei Erfolg
    \subsubitem HTTP-Status 201 im HTTP-Header, sowie die erstellte Idee als JSON im HTTP-Body
    \subitem Bei Fehler
    \subsubitem Fehlerrückmeldung
    \subitem Beispielantwort: siehe Anhang \ref{Anhang:Schnittstellen2}
\end{itemize}

\begin{itemize}
    \item HTTP-Methode: DELETE
    \subitem Relativer Pfad: \textit{/api/ideas/\{id\}}
    \subitem Antwort: HTTP-Status 200 bei Erfolg, ansonsten Fehlerrückmeldung
\end{itemize}

\begin{itemize}
    \item HTTP-Methode: PUT
    \subitem Relativer Pfad: \textit{/api/ideas/\{id\}}
    \subitem Mitzugebener HTTP-Body: Idee in JSON-Format (Syntax siehe GET-Methoden)
    \subitem Antwort: HTTP-Status 200 bei Erfolg, ansonsten Fehlerrückmeldung
\end{itemize}

\begin{itemize}
    \item Fehlerrückmeldung der API
    \subitem Beispielantwort:
    \begin{verbatim}
    {
        "timestamp": "2020-05-28 12:20:59.386",
        "status": 404,
        "error": "Not Found",
        "message": "Keine entsprechende Idee gefunden",
        "path": "/api/ideas/111"
    }
    \end{verbatim}
    \subitem Der Status der Antwort entspricht dem Status des HTTP-Header der Antwort.
\end{itemize}

\subsection{Login}
Um die GET- \textit{/api/ideas/} sowie die POST-, DELETE- und PUT- Aufrufe der API nutzen zu können, muss sich mit einem Benutzer der Rolle \textit{API\_USER} angemeldet werden.
Dies wird hier nicht genauer spezifiziert, da es keine Anforderung war.
Die Funktionalität wurde aber bereits in der Basis für zukünftige Erweiterungen hinzugefügt und kann mit einem REST-Client getestet werden.
Ein \textit{API\_USER} kann mit Hilfe des Skripts \textit{HelperScriptNoTests} erstellt werden.
Die restlichen GET-Methoden können ohne Authentifizierung aufgerufen werden. Diese entsprechen den Ansichten, die auch in der Benutzeroberfläche ohne Anmeldung angesehen werden können.

\subsection{Webschnittstelle}
Die restlichen Schnittstellen, die von der Benutzeroberfläche verwendet werden, verwenden das Format \textit{application/x-www-form-urlencoded}.
Jenes Format ermöglicht die native Verwendung von HTML-Forms ohne die Formulardaten per JavaScript umformen zu müssen.
Um diese Aufrufe testen zu können, können die Entwickleroptionen eines Webbrowsers oder ein REST-Client verwendet werden.
Dazu muss in den HTTP-Header des Requests der Cookie \textit{JSESSIONID} und das \textit{X-CSRF-TOKEN} eingefügt werden.
Diese können dem Browser in den Entwickleroptionen nach einem Login entnommen werden.