%!TEX root = ../Thesis.tex


\section{Fachkonzept \textcolor{blue}{[Jonathan Brockhausen]}}
\label{Technologien}

Im Folgenden werden die im Projekt verwendeten Technologien aufgeführt.

\subsection{Technologien}

\subsubsection*{Java 11}
Zum Ziele der größten Kompatibilität mit den weiteren Technologien haben wir uns für Version 11 des Java Development Kits als grundlegende Java-Version entschieden. Unter Java 11 können wir die ordnungsgemäße Funktionalität unserer Software garantieren, ein Update auf neuere Versionen kann zu Fehlern mit den Dependencies führen.
\subsubsection*{Maven (Wrapper)}
Um die notwendigen Dependencies des Projekts bereitzustellen, nutzen wir Maven mit einem Wrapper. Maven erlaubt in der POM.XML-Datei die einfache und übersichtliche Verwaltung von Dependencies. Die wichtigsten Dependencies sind im nächsten Überabschnitt aufgeführt. Der Wrapper unterstützt das Kompilieren des Projekts, da vorher Maven geladen wird und somit keine Dependecies manuell installiert werden müssen; sie werden beim Ausführen des Wrappers automatisch geladen.
\subsubsection*{GitHub}
GitHub ist die populärste und weitverbreiteste Plattform für Kollaboration auf Basis der Open-Source-Versionsverwaltung Git. Da alle drei Gruppenmitglieder mit der Plattform ansatzweise vertraut sind und die Integration in die verwendete IDE problemlos möglich ist, ist GitHub die ideale Plattform für das Projekt. Durch die GitHub CI-Tools wird die Funktionalität der Programmbestandteile automatisiert mit jedem Commit sichergestellt. Diese Funktion ist in \cref{Arbeitskonzept} näher dargestellt.
\subsubsection*{OpenProject, Teams, Telegram}
Für das Projektmanagement wurde OpenProject genutzt, eine Open-Source webbasierte Projektmanagement~Suite. Auf OpenProject wird in \autoref{Arbeitsaufteilung} weiter eingegangen.
Für die Kommunikation während des Projekts wurden Microsoft Teams für die Synchronisations-Calls und Pair Programming und Telegram für Messaging genutzt.

\subsection{Frameworks}
%
\subsubsection*{Springboot}
Springboot wird genutzt um einen Tomcat-Webserver zu starten und unseren Programmcode darin zu deployen. Es ist ein in der Branche übliches Framework, um die Produktion von Enterprise-Web-Anwendungen zu vereinfachen. Die bereitgestellte Standardkonfiguration bietet eine gut auf die Bedürfnisse der Anwendung anpassbare Grundlage. Dazu wurden in den \glqq{}application.properties\grqq{} unter anderem die Konfiguration der Programmbezeichnung, Port und Datenbankanbindung vorgenommen.
\subsubsection*{Thymeleaf}
Thymeleaf rendert HTML-Dateien. Das Framework erlaubt es, Informationen im Frontend einfach anzeigen und gleichzeitig ordentlich zu implementieren. Mithilfe von Thymeleaf vermeiden wir an einigen Stellen Konflikte oder Umwege. Dadurch, dass Java-Objekte und -Logik direkt in die HTML-Dateien eingebunden werden können und Thymeleaf daraus sichere Frontend-Seiten rendert, fließen Front- und Backend zwar in der Programmierung dichter zusammen, sind jedoch im Betrieb sauber getrennt.
Weiterhin ermöglicht Thymeleaf Integration mit Spring und dessen Security-Tools.

\subsection{Datenbank}
Als relationale Datenbank wurde eine h2-Datenbank verwendet. Wir nutzen JPA um die SQL-Tabellen mit Java-Klassen und -Objekten zu erzeugen. Durch die Verwendung von JPA wird vollständig auf die Erstellung von manuellen SQL-Queries verzichtet. Die h2-Datenbank ist dateibasiert und dadurch leichtgewichtiger und einfacher zu betreiben als eine server-basierte MSSQL- oder MySQL-Datenbank.