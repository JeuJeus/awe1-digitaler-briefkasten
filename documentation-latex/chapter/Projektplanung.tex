%!TEX root = ../Thesis.tex
%! Author = JB
%! Date = 22.05.2020


\section{Projektplanung \textcolor{blue}{[Jonathan Brockhausen]}}

\subsection{Projektstrukturplan}
Das Projekt wurde von uns in vier Phasen aufgeteilt, \textit{Vorbereitung, Implementierung, Dokumentation \& Tests} und \textit{Abschluss.}
Der Projektstrukturplan ist im Anhang auf Seite \pageref{PSP} dargestellt.

\subsection{Soll-Ist-Vergleich}
Der vor dem Projekt von uns festgelegte und in der Präsentation des Fachkonzepts vermittelte Soll-Zustand ist die vollständige Umsetzung der Muss-Features und die in der angegebenen Reihenfolge begonnene Umsetzung der Kann-Features.
Im Anhang auf Seite \pageref{PM_SOLLIST} ist die Übersicht der Features dargestellt. Alle Soll-Features wurden anforderungsgemäß umgesetzt. Die Umsetzung der Kann-Features wurde gemäß der im Fachkonzept vorgestellten Priorisierung begonnen. Im Einklang mit dem gesamten Projekt wurde bei allen Features darauf geachtet, dass sie gut erweiter- und wartbar sind.\\
Die Programmierung einer REST-API wurde von uns als wichtigstes Kann-Feature priorisiert. Besonders im Unternehmenskontext kommen oft Schnittstellen zwischen sehr verschiedenen Programmen vor. Mit der Verwendung einer REST-API haben wir eine in gewissen Maßen standardisierte Schnittstelle, die durch das universelle Rückgabeformat JSON eine Anbindung im Unternehmen unterstützt. \\
Als zweites Kann-Feature haben wir ein Kontaktformular umgesetzt. Das Kontaktformular bietet für Benutzer und Administratoren gleichzeitig eine an das System angebundene Anlaufstelle für Problemmeldungen und Anfragen. Die Nachrichten laufen im Administrator-Interface auf und sind dort für alle Administratoren sicht- und bearbeitbar.\\
Der Administrator war das dritte Kann-Feature welches von Anfang an hoch priorisiert war und früh im Programm umgesetzt wurde. Neben der Benutzerverwaltung können Administratoren Spezialisten anlegen und weitere Einträge in den Vorlauftabellen anlegen. Außerdem kommen die oben genannten Kontaktnachrichten im Administrator-Interface an. Der modulare Aufbau des Administrator-Interfaces macht es einerseits übersichtlich für den Nutzer und andererseits gut erweiterbar um weitere Funktionen.\\
Die übrigen Kann-Features wurden zunächst nicht implementiert. Das Projekt kann jedoch um diese Features erweitert werden ohne bestehende Logik zu sehr verändern zu müssen. Eine Mail-Server-Anbindung wäre ein sinnvoller nächster Schritt der die Einbindung einiger weiterer Features und die Erweiterung von bestehenden Features ermöglicht (beispielsweise das Kontaktformular).

\subsection{Arbeitsaufteilung}
\label{Arbeitsaufteilung}

Für die Arbeitsaufteilung wurden die regelmäßigen Synchronisations-Calls genutzt. Die anstehenden Aufgaben wurden in Arbeitspakete aufgeteilt und gemeinsam im Team verteilt. Hierbei wurden persönliche Fähigkeiten und Vertrautheit mit der speziellen Code-Stelle besonders in Betracht gezogen. Aufgrund der oben erwähnten All-Hands-On-Deck-Methode wurde sichergestellt, dass alle Teammitglieder informiert waren, wer an welcher Stelle arbeitet und somit Konflikte im Code vermieden.
Für das Festhalten der Arbeitspakete und der individuellen Fortschritte wurde das Projektmanagement-Tool OpenProject verwendet. Der Umfang des Projekts lässt über die Sinnhaftigkeit eines dedizierten Projektmanagementtools sicherlich streiten, aber unter dem Strich konnte so deutlich besser eine Struktur in die Arbeitsaufteilung gebracht werden. \\