\usepackage{cleveref}

%!TEX root = ../Thesis.tex
%! Author = PR
%! Date = 27.05.2020


\section{Schnittstellen}

Die Anwendung ist über eine REST-API erreichbar. Alle Antworten sind im JSON-Format. Die Schnittstelle stellt folgende Services bereit:
\begin{itemize}
    \item HTTP Methode: GET
    \subitem Relativer Pfad: /api/ideas/
    \subitem Antwort: Array, das alle Ideen (Proukt-/ Interne Ideen) beinhaltet.
    \subitem Beispielantwort: siehe Anhang \cref{Anhang_Schnittstellen_1}
\end{itemize}

\begin{itemize}
    \item HTTP Methode: GET
    \subitem Relativer Pfad: /api/ideas/{id}
    \subitem Antwort: Idee (Proukt-/ Interne Idee) in JSON-Format
    \subitem Beispielantwort: siehe Anhang \cref{Anhang_Schnittstellen_2}
\end{itemize}

\begin{itemize}
    \item HTTP Methode: GET
    \subitem Relativer Pfad: /api/ideas/title/{title}
    \subitem Antwort: Idee (Proukt-/ Interne Idee) in JSON-Format
    \subitem Beispielantwort: siehe Anhang \cref{Anhang_Schnittstellen_2}
\end{itemize}

\begin{itemize}
    \item HTTP Methode: GET
    \subitem Relativer Pfad: /api/ideas/submitted
    \subitem Antwort: Array, das alle Ideen (Proukt-/ Interne Ideen), die veröffentlicht sind, beinhaltet.
    \subitem Beispielantwort: siehe Anhang \cref{Anhang_Schnittstellen_1}
\end{itemize}

\begin{itemize}
    \item HTTP Methode: POST
    \subitem Relativer Pfad: /api/ideas/
    \subitem Mitzugebener HTTP-Body: Idee in JSON-Format (Syntax siehe GET-Methoden)
    \subitem Antwort: HTTP-Status 200 bei Erfolg, ansonsten Fehlerrückmeldung
    \subitem Beispielantwort: todo
\end{itemize}

\begin{itemize}
    \item HTTP Methode: DELETE
    \subitem Relativer Pfad: /api/ideas/{id}
    \subitem Antwort: HTTP-Status 200 bei Erfolg, ansonsten Fehlerrückmeldung
    \subitem Beispielantwort: todo
\end{itemize}

\begin{itemize}
    \item HTTP Methode: PUT
    \subitem Relativer Pfad: /api/ideas/{id}
    \subitem Mitzugebener HTTP-Body: Idee in JSON-Format (Syntax siehe GET-Methoden)
    \subitem Antwort: HTTP-Status 200 bei Erfolg, ansonsten Fehlerrückmeldung
    \subitem Beispielantwort: todo
\end{itemize}

\begin{itemize}
    \item Fehlerrückmeldung der API
    \subitem Beispielantwort:
    \begin{verbatim}
    {
        "timestamp": "2020-05-28 12:20:59.386",
        "status": 404,
        "error": "Not Found",
        "message": "Keine entsprechende Idee gefunden",
        "path": "/api/ideas/111"
    }
    \end{verbatim}
    \subitem Der Status der Antwort entspricht dem Status des HTTP-Header der Anwort.
\end{itemize}
