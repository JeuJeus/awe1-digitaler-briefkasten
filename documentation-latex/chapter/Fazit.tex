%!TEX root = ../Thesis.tex
%! Author = PR
%! Date = 27.05.2020


\section{Schlussbetrachtung \textcolor{blue}{[Philipp Röring]}}

\subsection{Bewertung}
In Anhang~\ref{PM_SOLLIST} ist zu sehen, dass alle Muss-Features der Anwendung implementiert wurden. Die Kann-Features wurden gemäß der
vorgestellten Priorisierung implementiert. Es gab somit bezüglich der Implementierung keine Abweichungen von der Projektplanung.
\\
Die drei Entwickler der Anwendung haben die Programmbestandteile aufgeteilt und jeweils einzeln implementiert. Jedoch wurde bei Unklarheiten mehrmals zusammen nach Lösungen gesucht. Auch im Rahmen der
Qualitätssicherung wurde der Quelltext gemeinsam überprüft und besprochen. Unabhängig davon stand es jedem Entwickler frei, nach Absprache den Quelltext eines Anderen zu
überarbeiten, wenn er Fehler bzw. Unschönheiten in diesem gefunden hat. Die Zusammenarbeit wird allgemein als sehr gut eingeschätzt. Lediglich die rein digitale Kommunikation
aufgrund der aktuellen Covid-19 Situation hat die Zusammenarbeit in geringem Maße erschwert. Die Entwickler sind sich jedoch einig, dass dies nicht die Qualität
der erstellten Anwendung gesenkt hat.
\\
Durch den sehr frühen Beginn der Entwicklung konnten die Laufzeit des Gesamtprojektes sowie die einzeln vergebenen Deadlines
für die Entwickler mühelos eingehalten werden. Auch das Aneignen von Know-How über Spring, Spring-boot und Thymeleaf konnte
in die Projektlaufzeit integriert werden.
\\
In den letzten Releases der Anwendung liefen alle Tests, die automatischen sowie die manuellen GUI-Tests, fehlerfrei durch. Es folgt eine kurze
eigene Bewertung der Qualität der Anwendung (1-10 Punkte).

\begin{longtable}{|p{0.15\textwidth}|p{0.15\textwidth}|p{0.1\textwidth}|p{0.15\textwidth}|p{0.15\textwidth}|p{0.15\textwidth}|}
    \caption{Bewertung der Anwendung}\\
    \hline
    Änderbarkeit & Benutzbarkeit & Effizienz & Funktionalität & Übertragbarkeit & Zuverlässigkeit\\
    \hline
    8 & 8 & 6 & 8 & 10 & 9 \\
    \hline
\end{longtable}

Die etwas niedrige Effizienz im Gegensatz zu sehr hoher Übertragbarkeit ist auf die Java Programmiersprache zurückzuführen.
Die Bewertung der Zuverlässigkeit wurde nach dem ständigen Bestehen der Tests sowie keinem Aufkommen von Programmabbrüchen bewertet.
Die hohe Änderbarkeit resultiert aus strukturiertem Quelltext, bei dem sich an die Standard-Struktur von Spring Projekten gehalten wurde.
Die Funktionalität und Benutzbarkeit wurde von Dritt-Testern bewertet.

\subsection{Fazit}
Der Soll-Ist Vergleich hat gezeigt, dass das Entwicklerteam gut zusammenarbeiten kann und für weitere Projekte bestens geeignet ist.
Es wäre allerdings von Vorteil, wenn die Entwickler für weitere Projekte eine Vorlaufzeit bekommen um sich in benötigte
Technologien einzuarbeiten. Darüber hinaus wäre eine persönliche Kommunikation vorteilhaft.







